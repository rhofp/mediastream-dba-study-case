\documentclass{article}
\usepackage[utf8]{inputenc}
\usepackage[spanish]{babel}
\usepackage{graphicx}
\usepackage{anysize}
\usepackage{fancyhdr} 
\usepackage[export]{adjustbox}
\usepackage{titlesec}
\usepackage{enumitem}

% \usepackage{hyperref}
% \usepackage{float}
% \usepackage{tabu}

% Izquierda, derecha, arriba, abajo
\marginsize{2cm}{2cm}{1.2cm}{1.5cm} 
\renewcommand{\familydefault}{\sfdefault}
\decimalpoint%

\graphicspath{{assets/}}

\setlength{\parindent}{0in}
\titleformat*{\section}{\large\bfseries}

\newcommand{\materia}{BDA}
\newcommand{\clave}{2929}
\newcommand{\profesor}{Ing. Rodriguez Campos \textsc{Jorge Alberto}}
\newcommand{\grupo}{1}
\newcommand{\semestre}{2021-1}

\newcommand{\alumno}{Francisco Pablo \textsc{Rodrigo}}

\newcommand{\actividad}{Práctica 01}
\newcommand{\titulo}{Instalación del sistema operativo}

\newcommand{\fechaEntrega}{29 de septiembre de 2020}

%%%%%%%%%%%%%%%%%%%% ENCABEZADO %%%%%%%%%%%%%%%%%%%%%%%%%%%%
\pagestyle{fancy}
\fancyhf{}
\renewcommand{\headrulewidth}{0pt}
\fancyhead[R]{% Left header
    \begin{tabular}{l}
        \materia \\ 
        \actividad%
    \end{tabular}
    \,% Space
    \rule[-1.75\baselineskip]{0pt}{0pt}
    % Strut to ensure a 1/4 \baselineskip between image and header rule
    \includegraphics[height=3\baselineskip,valign=c]{unam}
}
\setlength{\headsep}{0.3in}


\begin{document}
%%%%%%%%%%%%%%%%%%% DATOS PORTADA %%%%%%%%%%%%%%%%%%%%%%%%
\thispagestyle{empty}
\begin{minipage}[t][5cm][t]{0.2\linewidth}
    \includegraphics[width=2.5cm]{unam.jpg}
    \vspace{10cm}

    \includegraphics[width=2.5cm]{fiblack}
\end{minipage}
\begin{minipage}[t]{0.7\linewidth}
    \vspace{-2.5cm}
    \LARGE{\textbf{Universidad Nacional Autónoma de México}}\\
    \Large{\textbf{Facultad de Ingeniería}} \\

    \large{\semestre}\\[2cm]

    \large{\textbf{\materia (\clave)}}\\
    \large{\textbf{Gpo: 1}}\\[5mm]
    \large{\textbf{Profesor:} \profesor}\\ [1.5cm]
    \begin{center}
        \LARGE{\textbf{\actividad}}\\
        \LARGE{\textbf{\titulo}}\\
    \end{center}

    \vspace{3.3cm}

    \large{\textbf{Alumno:} \alumno} \\[1.5cm]

    \begin{flushright}
        \fechaEntrega%
    \end{flushright}
\end{minipage}

\newpage
%%%%%%%%%%%%%%%%%%% CONTENIDO %%%%%%%%%%%%%%%%%%%%%%%%

\section*{Introducción}
% Hablar de Linux, Oracle Linux, distribuciones, comandos de admin, etc.
En esta práctica se realizará la instalación del sistema operativo Oracle
Linux de forma nativa. Es recomendable tener nociones básicas de sistemas
operativos (particiones, cargador de arranque, etc.) ya que esto facilitará el 
seguimiento de esta práctica, además, evitará posibles errores que podrían 
costarnos la perdida de nuestro sistema operativo actual (ya sea Windows o 
Linux) y peor aún de nuestra información (es altamente recomendable realizar un
respaldo). Esta práctica explora un Gnu/Linux más real, es decir, una vez 
instalado el sistema operativo iniciaremos sin una interfaz gráfica, se
realizarán configuraciones del grub, conectividad a internet, etc. y finalmente
se podremos instalar nuestro entorno de escritorio favorito, se recomienda Mate
pero adicional a ello agregaré \textit{i3}, un gestor de ventanas muy útil.

\section*{Objetivos}
El objetivo de esta práctica es realizar las actividades necesarias para 
instalar una distribución GNU/Linux sobre la cual se hará la instalación de una
base de datos Oracle en prácticas posteriores. Las instrucciones que se 
describen a continuación ilustran los pasos requeridos para instalar un sistema
operativo Oracle Linux 7.7

% \section*{Desarrollo}

\section*{C1. Respuestas del cuestionario previo}

\begin{enumerate}[label=\protect\textbf{\Alph*.}]
    \item \textbf{Investigar el concepto de Oracle Unbreakable Enterprise 
        Kernel (UEK)}\\
        Es un \textit{kernel} construido por Oracle que se enfoca en el 
        rendimiento, la estabilidad y mínimo de puertos traseros
        (\textit{backports}) para lo cual mantiene el código fuente principal 
        tanto práctico como sea posible.

    \item \textbf{Características y diferencias entre el concepto RedHat 
        SystemD Targets y los llamados “Run Levels” que se empleaban en 
        versiones anteriores de RedHat y Oracle Linux}\\
        Los \textit{Run levels} pertenecen a \textit{SysV} (casi en desuso) y
        proveían la de la habiliad para iniciar el sistema en diferentes modos 
        para diferenes propósitos, dependiendo del nivel en el que se ejecutará
        la máquina. A partir de la versión 7 de RHELL se incorpora 
        \textit{systemd} y por supuesto los \textit{systemd target unit} que
        permiten iniciar el sistema con solo los serivico que se indiquen para 
        determinado propósito, una idea bastante similar a los ``Run levels'',
        pero más flexible, ya que son run levels están perfectamente definidos.
        El nivel 0 corresponde al apagado, el nivel 1 al modo monousuario, 2 
        inicia en modo multiusuario, 4 personalizado por el administrador, 5
        modo interfaz gráfica y el 6 reinicio.

    \item \textbf{¿Qué relación existe entre RedHat y Oracle Enterprise 
        Linux?}\\
        Orracle Linux, así como Fedora, CentOS y algunos otros, son sistemas 
        derivados a partir de RedHat. Por otra parte, considero importante 
        resaltar que ambos siguen el mismo modelo de negocio, en el cual, el 
        código de los sistemas operativos es libre (como todos los sistemas 
        Linux), pero lo que se vende es el soporte técnico. Además, Oracle 
        Linux es un monstruo que crece más y más a tal punto de que ha comenzado
        a superar al O.S. de Readhat gracias a que se ofrece más estabilidad,
        flexibilidad, costos y servicios de nube (Cloud computing).
        % https://www.centroid.com/blog/oracle-linux-vs-red-hat-enterprise-
        % linux/
        % https://linuxhint.com/oracle_linux_vs_redhat/
        % https://pediaa.com/what-is-the-difference-between-oracle-
        % linux-and-red-hat-enterprise-linux/

    \item \textbf{Utilidad del archivo \texttt{/etc/inittab}}\\
        Es un archivo utilizado por \textit{System V} en el cual se configuran
        los \textit{run levels} en losque iniciará el equipo, así como los 
        programas, procesos y scripts que se cargarán en ese nivel cuando el 
        kernel se inicie.
\end{enumerate}

\section*{C2. Salida del script de validación}

\begin{center}
    \includegraphics[width=\linewidth]{ej-prac-01-validador}    
\end{center}

\section*{C3. Conclusiones}
Un usuario Linux nunca deja de aprender con cada distribución que instala, 
creo que es necesario ``descomponer para poder aprender'', es esta práctica es 
lo que hice, pero de un modo tan catastrófico. El principal aprendizaje que me 
lleve fue a la hora de utilizar mi partición \textit{boot}, al principio la
instale Oracle Linux junto a manjaro, pero hubo problemas de compatibilidad 
en un programa que utiliza el \textit{grub} para lanzar el sistema operativo,
llamado \textbf{linux}, el manjaro solo se llama \textbf{linux} y en Oracle 
Linux se llama \textbf{linux-efi}. Al final logré solucionarlo 
``temporalmente'', pero dicha solución nunca me convenció y cuando encontré un 
sistema operativo que me gusto más que manjaro (\textbf{Arco Linux}), decidí
intentar el proceso de nuevo y esta vez bajo la premisa de que la distribuciones
\textit{rolling release} no se llevan con Oracle Linux. Funcionó.\\

Otro conocimiento que me llevo de la práctica es la forma de instalar software 
en otras distribuciones de Linux, no me gusto tanto, me parece muy tardado y 
hay una cantidad enorme de software que no es soportado (o que las ligas están
rotas) por la comunidad RedHat y lo entiendo porque el uso no esta pensado para
usuario normales sino para empresas.\\

Por último, a manera de histórico me gustaría ``reportar'' un ``bug'' o falla
que al parecer solo sucedió en mi computadora (estuve intentando checar
el grupo de Gmail lo más posible),  al actualizar mi sistema operativo algunas
dependencias dejaron de funcionar, lo cual hizo que al reiniciar mi computadora
esta no encendiera, a continuación anexo la descripción del error y en la 
bibliografía anexo el sitio que me sirvió para solucionar el problema.

\begin{verbatim}
Failed to set MokListRt: Out of Resources
Could not create MokListRt: Out of Resources
Something has gone seriously wrong: import_mok_state() failed
: Out of Resources
\end{verbatim}

\renewcommand\refname{Bibliografía}
\begin{thebibliography}{99}
    \bibitem{RedBeard}
    RedBeard Abides, \textit{CentOS 7: Failed to set MokListRT: 
    Invalid Parameter} en \\
    \texttt{https://redbeardteaches.com/2019/01/31/centos-7-failed-to-set-\\
    moklistrt-invalid-parameter/}
    \bibitem{unixninja}
    Unix Ninja, \textit{Manually booting the Linux kernel from GRUB} en \\
    \texttt{https://www.unix-ninja.com/p/Manually\_booting\_the\_Linux\_kernel\_
    from\_GRUB}

\end{thebibliography}


\end{document}
