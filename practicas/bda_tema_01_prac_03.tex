\documentclass{article}
\usepackage[utf8]{inputenc}
\usepackage[spanish]{babel}
\usepackage{graphicx}
\usepackage{anysize}
\usepackage{fancyhdr} 
\usepackage[export]{adjustbox}
\usepackage{titlesec}
\usepackage{enumitem}

% \usepackage{hyperref}
% \usepackage{float}
% \usepackage{tabu}

% Izquierda, derecha, arriba, abajo
\marginsize{2cm}{2cm}{1.2cm}{1.5cm} 
\renewcommand{\familydefault}{\sfdefault}
\decimalpoint%

\graphicspath{{assets/}}

\setlength{\parindent}{0in}
\titleformat*{\section}{\large\bfseries}

\newcommand{\materia}{BDA}
\newcommand{\clave}{2929}
\newcommand{\profesor}{Ing. Rodriguez Campos \textsc{Jorge Alberto}}
\newcommand{\grupo}{1}
\newcommand{\semestre}{2021-1}

\newcommand{\alumno}{Francisco Pablo \textsc{Rodrigo}}

\newcommand{\actividad}{Ejercicio práctico 03}
\newcommand{\titulo}{Instalación y creación de una BD oracle}

\newcommand{\fechaEntrega}{1 de octubre de 2020}

%%%%%%%%%%%%%%%%%%%% ENCABEZADO %%%%%%%%%%%%%%%%%%%%%%%%%%%%
\pagestyle{fancy}
\fancyhf{}
\renewcommand{\headrulewidth}{0pt}
\fancyhead[R]{% Left header
    \begin{tabular}{l}
        \materia \\ 
        \actividad%
    \end{tabular}
    \,% Space
    \rule[-1.75\baselineskip]{0pt}{0pt}
    % Strut to ensure a 1/4 \baselineskip between image and header rule
    \includegraphics[height=3\baselineskip,valign=c]{unam}
}
\setlength{\headsep}{0.3in}


\begin{document}
%%%%%%%%%%%%%%%%%%% DATOS PORTADA %%%%%%%%%%%%%%%%%%%%%%%%
\thispagestyle{empty}
\begin{minipage}[t][5cm][t]{0.2\linewidth}
    \includegraphics[width=2.5cm]{unam.jpg}
    \vspace{10cm}

    \includegraphics[width=2.5cm]{fiblack}
\end{minipage}
\begin{minipage}[t]{0.7\linewidth}
    \vspace{-2.5cm}
    \LARGE{\textbf{Universidad Nacional Autónoma de México}}\\
    \Large{\textbf{Facultad de Ingeniería}} \\

    \large{\semestre}\\[2cm]

    \large{\textbf{\materia (\clave)}}\\
    \large{\textbf{Gpo: 1}}\\[5mm]
    \large{\textbf{Profesor:} \profesor}\\ [1.5cm]
    \begin{center}
        \LARGE{\textbf{\actividad}}\\
        \LARGE{\textbf{\titulo}}\\
    \end{center}

    \vspace{3.3cm}

    \large{\textbf{Alumno:} \alumno} \\[1.5cm]

    \begin{flushright}
        \fechaEntrega%
    \end{flushright}
\end{minipage}

\newpage
%%%%%%%%%%%%%%%%%%% CONTENIDO %%%%%%%%%%%%%%%%%%%%%%%%

% \section*{Introducción}
% TODO:- Hacer introducción

\section*{Objetivos}
Conocer y comprender el procedimiento básico necesario para configurar e 
instalar el software para posteriormente crear una base de datos Oracle
empleando herramientas gráficas como son \texttt{runInstaller}, 
\texttt{dbca}, \texttt{netca}.

% \section*{Desarrollo}

\section*{C1. Salida ejecución del validador}

\begin{center}
    \includegraphics[width=\linewidth]{ej-prac-03-validador}    
\end{center}

\newpage
\section*{C3. Comentarios y conclusiones}
% TODO:- Comentarios
Realizar la instalación de la base de datos en un sistema operativo que fue 
desarrollado por los creadores de la misma base de datos (Oracle) resultó muy 
sencillo pues todas las bibliotecas se encontraban en los repositorios 
oficiales, y no había que configurar archivos de más para que pudiera funcionar,
definitivamente es mejor así.\\ 

Por otra parte, creo que haber usado la base de datos Oracle anteriormente
(en la materia de bases de datos) hace que todo se facilite, por ejemplo,
el validador me arrojó el error que decía que ``El listener no estaba 
encendido'', lo encendí y seguía sin funcionar, por lo que recordé que mencionó 
que tardaba en refrescarse ese estado y me pude haber esperado a que se 
refrescará pero me fue más fácil ejecutar \texttt{lsnrctl reload} y después de 
esto todo funcionó muy bien.

\renewcommand\refname{Bibliografía}
\begin{thebibliography}{99}
    \bibitem{oracle} Oracle. \textit{Oracle Database Documentation} en 
        \texttt{https://docs.oracle.com/en/database/oracle/\\oracle-database/
        index.html}
\end{thebibliography}

\end{document}
