\documentclass{article}
\usepackage[utf8]{inputenc}
\usepackage[spanish]{babel}
\usepackage{graphicx}
\usepackage{anysize}
\usepackage{fancyhdr} 
\usepackage[export]{adjustbox}
\usepackage{titlesec}
\usepackage{enumitem}
\usepackage{listings}
\usepackage{xcolor}

% \usepackage{hyperref}
% \usepackage{float}
% \usepackage{tabu}

% Izquierda, derecha, arriba, abajo
\marginsize{2cm}{2cm}{1.2cm}{1.5cm} 
\renewcommand{\familydefault}{\sfdefault}
\decimalpoint%

\graphicspath{{assets/}}

\setlength{\parindent}{0in}
\titleformat*{\section}{\large\bfseries}

% Para insert código
\definecolor{codegreen}{rgb}{0,0.6,0}
\definecolor{codegray}{rgb}{0.5,0.5,0.5}
\definecolor{codepurple}{rgb}{0.58,0,0.82}
\definecolor{backcolour}{rgb}{1,1,1}

\lstdefinestyle{mystyle}{
    backgroundcolor=\color{backcolour},   
    commentstyle=\color{codegreen},
    keywordstyle=\color{magenta},
    numberstyle=\tiny\color{codegray},
    stringstyle=\color{codepurple},
    basicstyle=\ttfamily\footnotesize,
    breakatwhitespace=false,         
    breaklines=true,                 
    captionpos=b,                    
    keepspaces=true,                 
    % numbers=left,                    
    % numbersep=5pt,                  
    showspaces=false,                
    showstringspaces=false,
    showtabs=false,                  
    tabsize=2
}

\lstset{style=mystyle}

\newcommand{\materia}{BDA}
\newcommand{\clave}{2929}
\newcommand{\profesor}{Ing. Rodriguez Campos \textsc{Jorge Alberto}}
\newcommand{\grupo}{1}
\newcommand{\semestre}{2021-1}

\newcommand{\alumno}{Francisco Pablo \textsc{Rodrigo}}

\newcommand{\actividad}{Ejercicio práctico 04}
\newcommand{\titulo}{Privilegios de Administración, roles y mecanismos 
    de autenticación}

\newcommand{\fechaEntrega}{6 de octubre de 2020}

%%%%%%%%%%%%%%%%%%%% ENCABEZADO %%%%%%%%%%%%%%%%%%%%%%%%%%%%
\pagestyle{fancy}
\fancyhf{}
\renewcommand{\headrulewidth}{0pt}
\fancyhead[R]{% Left header
    \begin{tabular}{l}
        \materia \\ 
        \actividad%
    \end{tabular}
    \,% Space
    \rule[-1.75\baselineskip]{0pt}{0pt}
    % Strut to ensure a 1/4 \baselineskip between image and header rule
    \includegraphics[height=3\baselineskip,valign=c]{unam}
}
\setlength{\headsep}{0.3in}


\begin{document}
%%%%%%%%%%%%%%%%%%% DATOS PORTADA %%%%%%%%%%%%%%%%%%%%%%%%
\thispagestyle{empty}
\begin{minipage}[t][5cm][t]{0.2\linewidth}
    \includegraphics[width=2.5cm]{unam.jpg}
    \vspace{10cm}

    \includegraphics[width=2.5cm]{fiblack}
\end{minipage}
\begin{minipage}[t]{0.7\linewidth}
    \vspace{-2.5cm}
    \LARGE{\textbf{Universidad Nacional Autónoma de México}}\\
    \Large{\textbf{Facultad de Ingeniería}} \\

    \large{\semestre}\\[2cm]

    \large{\textbf{\materia (\clave)}}\\
    \large{\textbf{Gpo: 1}}\\[5mm]
    \large{\textbf{Profesor:} \profesor}\\ [1.5cm]
    \begin{center}
        \LARGE{\textbf{\actividad}}\\
        \LARGE{\textbf{\titulo}}\\
    \end{center}

    \vspace{3.3cm}

    \large{\textbf{Alumno:} \alumno} \\[1.5cm]

    \begin{flushright}
        \fechaEntrega%
    \end{flushright}
\end{minipage}

\newpage
%%%%%%%%%%%%%%%%%%% CONTENIDO %%%%%%%%%%%%%%%%%%%%%%%%

\section*{Objetivos}
Comprender y poner en práctica los conceptos referentes a los privilegios de 
administración así como los diferentes mecanismos de autenticación
que pueden emplearse en una base de datos.

\section*{C1. Código del programa script s-01}

\lstinputlisting[language=SQL]{ej-prac-04-codigo/s-01-version-bd.sql}

\newpage
\section*{C2. Código del programa script s-02}

\lstinputlisting[language=SQL]{ej-prac-04-codigo/s-02-roles.sql}

\newpage
\section*{C3. Código del programa script s-03}

\lstinputlisting[language=bash]{ej-prac-04-codigo/s-03-archivo-password.sh}

\textbf{Nota:} Se agregan los scripts \texttt{s-04-privs-admin.sql}(script s-04) 
y \texttt{s-05-schemas.sql} (script s-05), la práctica solo menciona que se 
incluya hasta el script s-03.\\

\section*{Código del programa script s-04}

\lstinputlisting[language=SQL]{ej-prac-04-codigo/s-04-privs-admin.sql}

\section*{Código del programa script s-05}

\lstinputlisting[language=SQL]{ej-prac-04-codigo/s-05-schemas.sql}

\section*{C4. Salida de ejecución del validador}
% \begin{center}
    \includegraphics[width=\linewidth]{ej-prac-04-validador}    
% \end{center}
\newpage
\section*{C5. Comentarios y conclusiones}

Me parece interesante que nosotros como alumnos tengamos que entender de manera
implícita como funciona el validador. Por ejemplo, al ejecutar el validador me
aparecía que muchas de las estructuras no habían podido crearse (por ejemplo
un ``header''), tarde algo de tiempo pero finalmente me di cuenta que mi último
script había iniciado sesión en un usuario que la práctica solicitaba pero no 
había tenido en cuenta que debía salir de ese usuario o en su defecto volverme 
a cambiar al usuario \textit{sys}, que fue lo que hice.\\

Por otra parte, esta práctica me permitió reafirmar las relaciones y/o eventos
que suceden al ingresar con determinado usuario y con determinado privilegio.
Además de verificar de manera tangible los archivos que se modifican las tirar
algunos comandos de sistema, por ejemplo para recuperar el archivo de 
contraseñas (comando \texttt{orapwd}, archivo \textit{orapwrfpbda1}).

\renewcommand\refname{Bibliografía y referencias}
\begin{thebibliography}{99}
    \bibitem{chartio} AJ Welch. \textit{How to Show All Oracle Database
    Privileges for a User} en \\
    \texttt{https://chartio.com/resources/tutorials/oracle-user-privileges-\\
    -how-to-show-all-privileges-for-a-user/}

    \bibitem{burleson} Burleson Consulting. \textit{Oracle tips } en 
        \texttt{http://www.dba-oracle.com/oracle\_news/}
\end{thebibliography}

\end{document}
