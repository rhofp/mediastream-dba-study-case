\documentclass{article}
\usepackage[utf8]{inputenc}
\usepackage[spanish]{babel}
\usepackage{graphicx}
\usepackage{anysize}
\usepackage{fancyhdr} 
\usepackage[export]{adjustbox}
\usepackage{titlesec}
\usepackage{enumitem}
\usepackage{listings}
\usepackage{xcolor}

% \usepackage{hyperref}
% \usepackage{float}
% \usepackage{tabu}

% Izquierda, derecha, arriba, abajo
\marginsize{2cm}{2cm}{1.2cm}{1.5cm} 
\renewcommand{\familydefault}{\sfdefault}
\decimalpoint%

\graphicspath{{assets/}}

\setlength{\parindent}{0in}
\titleformat*{\section}{\large\bfseries}

% Para insert código
\definecolor{codegreen}{rgb}{0,0.6,0}
\definecolor{codegray}{rgb}{0.5,0.5,0.5}
\definecolor{codepurple}{rgb}{0.58,0,0.82}
\definecolor{backcolour}{rgb}{1,1,1}

\lstdefinestyle{mystyle}{
    backgroundcolor=\color{backcolour},   
    commentstyle=\color{codegreen},
    keywordstyle=\color{magenta},
    numberstyle=\tiny\color{codegray},
    stringstyle=\color{codepurple},
    basicstyle=\ttfamily\footnotesize,
    breakatwhitespace=false,         
    breaklines=true,                 
    captionpos=b,                    
    keepspaces=true,                 
    % numbers=left,                    
    % numbersep=5pt,                  
    showspaces=false,                
    showstringspaces=false,
    showtabs=false,                  
    tabsize=2
}

\lstset{style=mystyle}

\newcommand{\materia}{BDA}
\newcommand{\clave}{2929}
\newcommand{\profesor}{Ing. Rodriguez Campos \textsc{Jorge Alberto}}
\newcommand{\grupo}{1}
\newcommand{\semestre}{2021-1}

\newcommand{\alumno}{Francisco Pablo \textsc{Rodrigo}}

\newcommand{\actividad}{Ejercicio práctico 02}
\newcommand{\titulo}{Programación en Shell y variables de entorno}

\newcommand{\fechaEntrega}{1 de octubre de 2020}

%%%%%%%%%%%%%%%%%%%% ENCABEZADO %%%%%%%%%%%%%%%%%%%%%%%%%%%%
\pagestyle{fancy}
\fancyhf{}
\renewcommand{\headrulewidth}{0pt}
\fancyhead[R]{% Left header
    \begin{tabular}{l}
        \materia \\ 
        \actividad%
    \end{tabular}
    \,% Space
    \rule[-1.75\baselineskip]{0pt}{0pt}
    % Strut to ensure a 1/4 \baselineskip between image and header rule
    \includegraphics[height=3\baselineskip,valign=c]{unam}
}
\setlength{\headsep}{0.3in}


\begin{document}
%%%%%%%%%%%%%%%%%%% DATOS PORTADA %%%%%%%%%%%%%%%%%%%%%%%%
\thispagestyle{empty}
\begin{minipage}[t][5cm][t]{0.2\linewidth}
    \includegraphics[width=2.5cm]{unam.jpg}
    \vspace{10cm}

    \includegraphics[width=2.5cm]{fiblack}
\end{minipage}
\begin{minipage}[t]{0.7\linewidth}
    \vspace{-2.5cm}
    \LARGE{\textbf{Universidad Nacional Autónoma de México}}\\
    \Large{\textbf{Facultad de Ingeniería}} \\

    \large{\semestre}\\[2cm]

    \large{\textbf{\materia (\clave)}}\\
    \large{\textbf{Gpo: 1}}\\[5mm]
    \large{\textbf{Profesor:} \profesor}\\ [1.5cm]
    \begin{center}
        \LARGE{\textbf{\actividad}}\\
        \LARGE{\textbf{\titulo}}\\
    \end{center}

    \vspace{3.3cm}

    \large{\textbf{Alumno:} \alumno} \\[1.5cm]

    \begin{flushright}
        \fechaEntrega%
    \end{flushright}
\end{minipage}

\newpage
%%%%%%%%%%%%%%%%%%% CONTENIDO %%%%%%%%%%%%%%%%%%%%%%%%

\section*{Objetivos}
Practicar los conceptos básicos y fundamentales de la programación Shell en 
específico GNU Bash (Bourne-again shell). Estos conceptos serán empleados
durante el curso como parte de las actividades de administración de bases de 
datos.

\section*{C1. Código fuente del programa}

\lstinputlisting[language=bash]{ej-prac-02-codigo/s-01-procesa-imagenes.sh}

\section*{C1. Salida ejecución del validador}

\begin{center}
    \includegraphics[width=\linewidth]{ej-prac-02-validador}    
\end{center}

\newpage
\section*{C3. Comentarios y conclusiones}

La programación en este lenguaje de shell scripting (\textbf{bash}) resulta 
tan contrastante con la verbosidad del lenguaje \textbf{Java}, que es al que a 
veces estamos acostumbrados. De hecho resulta curioso, porque prácticamente 
son polos opuestos. En \textit{shell scripting} tenemos utilizar muchas 
abreviaciones que en principio no podrían parecer lógicas, por ejemplo, 
\texttt{-d, -f, -z}, etc. En cambio en \textit{Java} tenemos que escribir cosas
tan raras como \texttt{System.out.println(``Hola mundo'')} (en extremo verboso).
No creo ningún lenguaje sea mejor que otro son simples diseños que  tienen su 
razón de ser.\\ 

Por otra parte, me parece muy interesante el hecho de automatizar tareas. De 
hecho una de las razones para decidir ser usuario Linux fue esa. Estoy al tanto
de que la automatización se podría hacer en los diferentes sistemas operativos,
sin embargo, creo que en Linux resula muy sencillo y hasta natural. Además 
es muy sencillo integrar las herramientas que vienen incluidas en el sistema
operativo, por ejemplo, en el ejercicio utilice \texttt{awk} para realizar 
la lista de archivos descargados.

\renewcommand\refname{Bibliografía y referencias}
\begin{thebibliography}{99}
    \bibitem{shell} Steve Parker. \textit{Shell Scripting Tutorial} en 
    \texttt{https://www.shellscript.sh/}

    \bibitem{wikibooks} Wikibooks. \textit{Bourne Shell Scripting} en 
        \texttt{https://en.wikibooks.org/wiki/Bourne\_Shell\_Scripting}
\end{thebibliography}

\end{document}
