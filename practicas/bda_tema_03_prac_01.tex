\documentclass{article}
\usepackage[utf8]{inputenc}
\usepackage[spanish]{babel}
\usepackage{graphicx}
\usepackage{anysize}
\usepackage{fancyhdr} 
\usepackage[export]{adjustbox}
\usepackage{titlesec}
\usepackage{enumitem}
\usepackage{listings}
\usepackage{xcolor}
\usepackage{array}
\usepackage{longtable}
\usepackage{multicol}
% \usepackage{hyperref}
% \usepackage{float}
% \usepackage{tabu}

% Izquierda, derecha, arriba, abajo
\marginsize{2cm}{2cm}{1.2cm}{1.5cm} 
\renewcommand{\familydefault}{\sfdefault}
\decimalpoint%

\graphicspath{{assets/}}

\setlength{\parindent}{0in}
\titleformat*{\section}{\large\bfseries}

% Para insert código
\definecolor{codegreen}{rgb}{0,0.6,0}
\definecolor{codegray}{rgb}{0.5,0.5,0.5}
\definecolor{codepurple}{rgb}{0.58,0,0.82}
\definecolor{backcolour}{rgb}{1,1,1}

\usepackage{textcomp}
\lstset{upquote=true}
\lstdefinestyle{mystyle}{
    backgroundcolor=\color{backcolour},   
    commentstyle=\color{codegreen},
    keywordstyle=\color{magenta},
    numberstyle=\tiny\color{codegray},
    stringstyle=\color{codepurple},
    basicstyle=\ttfamily\footnotesize,
    breakatwhitespace=false,         
    breaklines=true,                 
    captionpos=b,                    
    keepspaces=true,                 
    % numbers=left,                    
    % numbersep=5pt,                  
    showspaces=false,                
    showstringspaces=false,
    showtabs=false,                  
    tabsize=2
}

\lstset{style=mystyle}

\newcommand{\materia}{BDA}
\newcommand{\clave}{2929}
\newcommand{\profesor}{Ing. Rodriguez Campos \textsc{Jorge Alberto}}
\newcommand{\grupo}{1}
\newcommand{\semestre}{2021-1}

\newcommand{\alumno}{Francisco Pablo \textsc{Rodrigo}}

\newcommand{\actividad}{Tema 03 \\ Ejercicio práctico 01}
\newcommand{\titulo}{Administración de parámetros}

\newcommand{\fechaEntrega}{5 de noviembre de 2020}

%%%%%%%%%%%%%%%%%%%% ENCABEZADO %%%%%%%%%%%%%%%%%%%%%%%%%%%%
\pagestyle{fancy}
\fancyhf{}
\renewcommand{\headrulewidth}{0pt}
\fancyhead[R]{% Left header
    {\renewcommand*{\arraystretch}{1}
    \begin{tabular}{l}
        \materia \\ 
        \actividad%
    \end{tabular}}
    \,% Space
    \rule[-1.75\baselineskip]{0pt}{0pt}
    % Strut to ensure a 1/4 \baselineskip between image and header rule
    \includegraphics[height=3\baselineskip,valign=c]{unam}
}
\setlength{\headsep}{0.3in}


\begin{document}
%%%%%%%%%%%%%%%%%%% DATOS PORTADA %%%%%%%%%%%%%%%%%%%%%%%%
\thispagestyle{empty}
\begin{minipage}[t][5cm][t]{0.2\linewidth}
    \includegraphics[width=2.5cm]{unam.jpg}
    \vspace{10cm}

    \includegraphics[width=2.5cm]{fiblack}
\end{minipage}
\begin{minipage}[t]{0.7\linewidth}
    \vspace{-2.5cm}
    \LARGE{\textbf{Universidad Nacional Autónoma de México}}\\
    \Large{\textbf{Facultad de Ingeniería}} \\

    \large{\semestre}\\[2cm]

    \large{\textbf{\materia (\clave)}}\\
    \large{\textbf{Gpo: 1}}\\[5mm]
    \large{\textbf{Profesor:} \profesor}\\ [1.5cm]
    \begin{center}
        \LARGE{\textbf{\actividad}}\\
        \LARGE{\textbf{\titulo}}\\
    \end{center}

    \vspace{3.3cm}

    \large{\textbf{Alumno:} \alumno} \\[1.5cm]

    \begin{flushright}
        \fechaEntrega%
    \end{flushright}
\end{minipage}

\newpage
%%%%%%%%%%%%%%%%%%% CONTENIDO %%%%%%%%%%%%%%%%%%%%%%%%

\section*{Objetivos}
Comprender las características, acciones realizadas y comportamiento de las 
diferentes etapas y modos que tiene una base de datos tanto para
iniciar como para detener una instancia.

\section*{C1. Tabla de respuestas}

% \begin{table}[h!]
% \setlength{\tabcolsep}{10pt}
% \begin{tabular}{
\begingroup
% \def\arraystretch{2}
\renewcommand*{\arraystretch}{2}
\begin{longtable}[c]{
    |>{\centering}p{0.1\textwidth}
    |>{\centering}p{0.1\textwidth}
    |>{\arraybackslash}p{0.7\textwidth}|
}
\hline
Pregunta & Respuesta (incisco) & Explicación/Justificación \\ 
\hline
P01 & C & 
Escribir la sentencia \texttt{alter database close;} es la forma más sútil 
de cerrar una base de datos, es equivalente a realizar un \texttt{shutdown}.
Las instrucciones esperan hasta que todas las sesiones sean cerradas para poder
detener la instancia.
\\
\hline
P02 & E & 
El comando \texttt{shutdown} es una de las formas más sútles para detener una
instancia, el comando espera hasta que todas las sesiones sean cerradas, en este
caso, la sesión en \textit{T2} permanece abierta y por lo tanto en \textit{T1}
se bloquea.
\\
\hline
P03 & B & 
Al liberar la sesión en \textit{T2} se desbloquea \textit{T1} y por lo tanto
se puede proceder a detener la instancia.
\\
\hline
P04 & A & 
La instancia habia sido detenida previamente, por ello es posible iniciarla con
startup en el modo que nosotros que indiquemos, de lo contrario deberíamos 
ocupar \texttt{alter database}.
\\
\hline
P05 & C & 
La base de datos se inicio en modo \texttt{nomount}, lo que significa que 
simplemente se inicio la instancia, pero para poder autentificar a usuarios
no administradores es necesio que la base de datos se encuentre en modo open,
ya que de esta manera se podrá consultar el diccionario de datos.
\\
\hline
P06 & C & 
Con la instrucción \texttt{alter database} nos podemos cambiar de modo y por
lo tanto es posible abrir la base de datos.
\\
\hline
P07 & A & 
Al tener la base de datos en modo \texttt{open} se puede autentificar al usuario
empleando el diccionario de datos y por lo tanto se puede logear si sus 
credenciales son correctas. Además, es posible realizar transacciones porque
podemos acceder a los redo logs y a los data files.
\\
\hline
P08 & E & 
Ninguna de las sesiones actuales esta realizando ninguna transacción por lo 
tanto, es posible cerrar la base de datos.
\\
\hline
P09 & C & 
La instancia ha sido detenida y por lo tanto no es posible crear nuevas 
transacciones.
\\
\hline
P10 & E & 
Por cada conjunto de instrucción DML se crea una transacción implícita si
no se específica lo contrario. En este caso la transacción en \texttt{12} falló
por completo, por lo tanto, no tiene nada que ver con esta nueva transacción
que se ejecuta en una base de datos abierta.
\\
\hline
P11 & D & 
\texttt{shutdown immediate} es una de las formas más ``rudas'' para detener
una instancia de base de datos y las transacciones de los usuarios se les 
aplicará \textit{rollback}.
\\
\hline
P12 & A & 
Cómo en \texttt{T15} se ejecutó \texttt{shutdown immediate} y esto produjó
un \textit{rollback}, ahora tenemos 0 registros.
\\
\hline
P13 & E & 
No hay transacciones ejecutándose en ninguna termial, por lo tanto, se
detiene la base de datos con éxito.
\\
\hline
% \end{tabular}
\end{longtable}
% \end{table}  
\endgroup 
\section*{C2. Contenido del archivo 
\texttt{/unam-bda/ejercicios-practicos/t0301/eventos.log}}
\begin{multicols}{2}
\textbf{T1}
\lstinputlisting[language=Bash]
    {tema03-ej-prac-01-codigo/eventos-1.log}
\columnbreak%
\textbf{T2}
\lstinputlisting[language=Bash]
    {tema03-ej-prac-01-codigo/eventos-2.log}    
\end{multicols}

\section*{Comentarios y conclusiones}

Está ejercicio nos sirvió para entender los modos de operación de una base de 
datos de manera práctica. Además, se ilustraron algunos casos que nos podríamos
llegar a encontrar en la práctica profesional. Por ejemplo, ¿qué pasa si un dba
quiere detener la instancia pero no quiere que las transacciones de otros
usuarios se pierdan? Ahora sabemos que la respuesta es que debe hacer un 
\texttt{shutdown transaccional}.
También se revisaron otros comportamiento que a veces nos podrían parecer un
tanto erráticos, por ejemplo, el hecho de que se bloquee una base de datos 
cuando la querramos apagar, o el hecho de que se haga rollback o commit cuando 
querramos detener la instancia. 
\renewcommand\refname{Bibliografía y referencias}
\begin{thebibliography}{99}
    \bibitem{burleson} Burleson Consulting. \textit{Oracle tips } en 
    \texttt{http://www.dba-oracle.com/oracle\_news/}
    \bibitem{oracle} Oracle Help Center. \textit{Spool on/off}. Technical 
    Reference en 
    \texttt{https://docs.oracle.com/
    cd/E57185\_01/ESBTR/maxl\_commands\_spool.html}
\end{thebibliography}

\end{document}
